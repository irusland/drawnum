\documentclass{article}
 
%Russian-specific packages
%--------------------------------------
\usepackage[T2A]{fontenc}
\usepackage[utf8]{inputenc}
\usepackage[russian]{babel}
%--------------------------------------
 
%Hyphenation rules
%--------------------------------------
\usepackage{hyphenat}
\hyphenation{ма-те-ма-ти-ка вос-ста-нав-ли-вать}
%--------------------------------------
 
\title{Классификатор нарисованных вручную цифр на основе сверточных нейронных сетей}
\author{Сиражетдинов Руслан ФТ-401}
 
\begin{document}
\maketitle
\newpage

\tableofcontents
\newpage

\begin{abstract}
  Это вводный абзац в начале документа.
\end{abstract}
 
\section{Введение}
Распознование изображений компьютером является задачей компьютерного зрения и является актуальной в наше время цифровизации и автоматизации различных процессов. Одной из таких задач является распознование рукописных записей, для её решения можно применять различные алгоритмы, в том числе и нейронные сети, которые широко распространены, применяются и очень хорошо справляются с обработкой и классификацией изображений.

Для решения задач этой области применяются различные технологии, одним из удобных инструментом для реализации поставленных задач является \verb|MatLab| - пакет прикладных программ для технических вычислений и создания графических интерфесов.

\verb|MatLab| для рещения задачи классификации был выбран ввиду того, что по приложению существует большое колличество документации и есть возможность установки доплнительных пакетов специфичных для этой области.
\newpage


\section{Архитектура проекта}
Проект состоит из трёх частей - 
\begin{itemize}
  \item Реестр данных для обучения - набор изображений с цифрами
  \item Сверточная нейронная сеть - модель классификатора
  \item Графический интерфейс - для ввода и вывода данных пользователя
\end{itemize}
\newpage


\section{Модель классификатора}
\newpage

\section{Обучение модели}
\newpage

\section{Создание графического интерфейса}
Кириллические символы также могут быть использованы в математическом режиме.
\newpage 

\section{Демонстрация работы приложения}
\newpage

\section{Заключение}
\newpage
 
\begin{equation}
  S_\textup{ис} = S_{123}
\end{equation}
 
\end{document}